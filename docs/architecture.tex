%& --translate-file latin2pl
\documentclass{article}

\usepackage{polski}
\usepackage{hyperref}
\usepackage[T1]{fontenc}
\usepackage[utf8]{inputenc}
\usepackage{listings}
\hypersetup{
colorlinks=true,
urlcolor=blue,
}
\setcounter{section}{-1}
\lstset{
basicstyle=\fontsize{6}{6}\selectfont\ttfamily,
}
\usepackage{graphicx}
\graphicspath{ {./images/} }
\usepackage[a4paper, total={6in, 8in}]{geometry}

\title{
Lekizapteki\\
\large architektura}
\author{Marcin Abramowicz \and Mateusz Danowski \and Dawid Jamka \and Tomasz Patyna}


\begin{document}
  \maketitle

  \section{Dziennik zmian}
  \begin{tabular}{|c|c|c|}
    Nr iteracji & Data & Opis zmian \\
    \hline
    1. & 25.03.2020 & Utworzenie dokumentu oraz jego pierwsza wersja. \\
    \hline
    1. & 29.04.2020 & Poprawki w endpointach i schema. \\
    \hline
    2. & 29.06.2020 & Współczynnik opłacalności w wyświetlanej tabeli oraz mała modyfikacja api. \\
    \hline
    2. & 01.06.2020 & Aktualizacja bazy danych.
  \end{tabular}

  \section{Ogólna struktura}
  Aplikacja składa się z frontendu WWW napisanego w
  \href{https://angular.io}{Angularze}, backendu napisanego w
  \href{https://spring.io}{Spring Framework} oraz bazy danych
  \href {https://www.h2database.com/html/main.html}{H2}.

  \section{Komunikacja}
  \subsection{Backend - Baza Danych}
  Backend komunikuje się z lokalną bazą danych dzięki
  \href{https://hibernate.org}{Hibernate} oraz
  \href{https://spring.io/projects/spring-data-jpa} {Spring Jpa}.
  Framework zajmuje się konstruowaniem zapytań oraz tworzeniem schematu bazy danych.

  \subsection{Frontend - Backend}
  Komunikacja jest synchroniczna i odbywa się protokołem HTTP, serwer dostarcza
  \href{https://en.wikipedia.org/wiki/Representational_state_transfer}{RESTful API},
  obsługując żądania w formacie
  \href{https://en.wikipedia.org/wiki/JSON}{json}.
  Serwer jest dostępny na ``Google cloud'' na porcie 7312
  natomiast frontend wykonuje zapytania za pomocą
  \href{https://angular.io/guide/http}{HttpClient'a} służącego do wykonywania zapytań http.
  Nie ma stosowanych żadnych form zabezpieczeń, żądania i odpowiedzi nie są szyfrowane, ani nie ma wymiany tokenów.
  API backendu jest publiczne, więc każdy dysponujący linkiem może z niego korzystać.
  Backend akceptuje jedynie zapytania w formacie \texttt{application/json}.
  Odpowiedzi również są jedynie w formacie \texttt{application/json}
  Oznacza to, że zapytania do backendu powinny być z nagłówkiem: \texttt{Accept: application/json},
  natomiast odpowiedzi będą z nagłówkiem: \texttt{Content-Type: application/json}.

  \subsubsection{Endpointy udostępniane przez backend}
  \noindent
  \begin{minipage}{.45\textwidth}
    \begin{lstlisting}
      - GET /lekizapteki/diseases

      Response:
      HTTP/1.1 200 (OK)
      Body:
      [{
      "id": Long,
      "name":"String"
      }]


      - GET /lekizapteki/medicines

      Parameters:
      diseaseId: Long (optional)

      Responses:
      HTTP/1.1 200 (OK)
      Body:
      [{
      "ean": "String",
      "name": "String",
      "dose": "String"
      }]

      HTTP/1.1 404 (Not Found)
      Message: "Nieprawidlowa jednostka chorobowa"

      - complex types definitions
      PricingDto: {
      "tradePrice": Decimal,
      "salePrice": Decimal,
      "retailPrice": Decimal,
      "totalFunding": Decimal,
      "chargeFactor": Decimal,
      "refund": Decimal,
      "isLumpSum": boolean,
      "isFree": boolean,
      "profitabilityRate": Decimal
      }

    \end{lstlisting}
  \end{minipage}\hfill
  \begin{minipage}{.45\textwidth}
    \begin{lstlisting}
      MedicineDto: {
      "ean": "String",
      "name": "String",
      "activeIngredient": String
      "dose": "String",
      "form": "String",
      "pricing": PricingDto
      }

      - GET /lekizapteki/medicines/identical

      Parameters:
      ean: String
      diseaseId: Long (optional)

      Responses:
      HTTP/1.1 200 (OK)
      Body:
      {
      "medicine": MedicineDto,
      "identicalMedicines: [MedicineDto]
      }

      HTTP/1.1 404 (Not Found)
      Message: "Nieprawidlowy numer EAN"

      HTTP/1.1 404 (Not Found)
      Message: "Lek o podanym numerze EAN nie jest
      przepisywany na wybrana jednostke chorobowa"
    \end{lstlisting}
  \end{minipage}

  \section{Frontend}
  Frontend jest dostępny na
  \href{http://students.mimuw.edu.pl:1237}{serwerze Students na porcie `1237`} i jest publiczny,
  każdy dysponujący linkiem może z niego korzystać.
  Użytkowanie nie wymaga zakładania konta ani autoryzacji.

  Frontend posiada web serwis oraz modele odpowiadające odbieranym danym z backendu.
  Do tworzenia elementów używane są wbudowane komponenty oraz odpowiednie komponenty ogólnodostępne w internecie.

  \subsection{Makiety}
  \subsubsection{Wybór choroby}
  \includegraphics[width=8cm]{images/lekizapteki-wybor-choroby}

  \subsubsection{Wybór leku za pomocą nazwy i numeru EAN}
  \includegraphics[width=8cm]{images/lekizapteki-wybor-leku-nazwa}
  \includegraphics[width=8cm]{images/lekizapteki-wybor-leku-ean}

  \subsubsection{Wyniki i rozwinięte szczegóły dla pozycji}
  \includegraphics[width=8cm]{images/lekizapteki-leki-identyczne}
  \includegraphics[width=8cm]{images/lekizapteki-leki-identyczne-rozwiniete}

  \section{Backend}
  Backend jest podzielony na dwa moduły: ``API'' i ``APP''.

  \subsection{Moduł API}
  Zawiera wystawione endpoint'y odpowiadające tym przedstawionym w punkcie 2.1.1.
  Są one w postaci interface'ów, klas z modelami oraz wyjątków rzucanych przy nieodpowiednich danych.

  \subsection{Moduł APP}
  Zawiera implementacje interface'ów z modułu API.
  Każdy interface jest implementowany przez kontroler.
  Kontrolery mają wstrzyknięte jedynie klasy służące do wykonywania pojedynczego endpointa (Command Pattern).

  Encje i relacje bazy danych są odwzorowane klasami z odpowiednimi adnotacjami.
  Zapytania do bazy danych odbywają się przez interface, implementowany przez framework,
  w którym nazwy metod są mapowane na zapytania do bazy danych.

  Dane z rozporządzeń są pobierane bezpośrednio przez aplikację w serwisie, który jest uruchamiany zaraz po powstaniu kontekstu.
  Może on pracować w dwóch trybach, które są konfigurowane w pliku \texttt{application.properties}.
  Pierwszym trybem jest pobieranie plików z podanego w konfiguracji linku,
  drugim jest pobieranie z lokalnego źródła, do którego ścieżka jest podana w konfiguracji.
  Parsuje on pliki z arkuszami Excel, waliduje, filtruje spełniające założenia serwisu oraz mapuje na encje w bazie danych.
  \section{Baza Danych}
  Baza danych jest \texttt{in-memory}, czyli jest tworzona przy tworzeniu kontekstu aplikacji.

  \subsection{Schemat}
  \noindent
  \begin{minipage}{.45\textwidth}
    \begin{lstlisting}
      CREATE TABLE DISEASE (
      id              IDENTITY    NOT NULL,
      name            TEXT        NOT NULL,

      CONSTRAINT disease_pk PRIMARY KEY (id)
      );

      CREATE TABLE INGREDIENT (
      id              IDENTITY    NOT NULL,
      name            TEXT        NOT NULL,

      dose_id         NUMERIC     NOT NULL,
      medicine_id     NUMERIC     NOT NULL,

      CONSTRAINT ingredient_pky PRIMARY KEY (id),

      FOREIGN KEY (medicine_id) REFERENCES MEDICINE (id),
      FOREIGN KEY (dose_id) REFERENCES DOSE (id)
      );

      CREATE TABLE DOSE (
      id              IDENTITY    NOT NULL,
      dose            NUMERIC     NOT NULL,

      CONSTRAINT dose_pk PRIMARY KEY (id)
      );

      CREATE TABLE FORM (
      id              IDENTITY    NOT NULL,
      name            TEXT        NOT NULL,

      CONSTRAINT form_pk PRIMARY KEY (id)
      );

      CREATE TABLE PACKAGE (
      id              IDENTITY    NOT NULL,
      quantity        NUMERIC     NOT NULL,

      CONSTRAINT package_pk PRIMARY KEY (id)
      );
    \end{lstlisting}
  \end{minipage}\hfill
  \begin{minipage}{.45\textwidth}
    \begin{lstlisting}
      CREATE TABLE PRICING (
      id              IDENTITY    NOT NULL,
      trade_price     DECIMAL     NOT NULL,
      sale_price      DECIMAL     NOT NULL,
      retail_price    DECIMAL     NOT NULL,
      total_funding   DECIMAL     NOT NULL,
      charge_factor   DECIMAL     NOT NULL,
      refund          DECIMAL     NOT NULL,
      is_lump_sum     BOOLEAN     NOT NULL,
      is_free         BOOLEAN     NOT NULL,

      CONSTRAINT pricing_pk PRIMARY KEY (id)
      );

      CREATE TABLE MEDICINE (
      id              IDENTITY    NOT NULL,
      ean             TEXT        NOT NULL,
      name            TEXT        NOT NULL,
      form_id         NUMERIC     NOT NULL,
      pricing_id      NUMERIC     NOT NULL,
      package_id      NUMERIC     NOT NULL,
      disease_id      NUMERIC     NOT NULL,

      CONSTRAINT medicine_pk PRIMARY KEY (id)

      FOREIGN KEY (form_id) REFERENCES FORM (id),
      FOREIGN KEY (pricing_id) REFERENCES PRICING (id),
      FOREIGN KEY (package_id) REFERENCES PACKAGE (id),
      FOREIGN KEY (disease_id) REFERENCES DISEASE (id),
      );
    \end{lstlisting}
  \end{minipage}

\end{document}
